\documentclass[conference]{IEEEtran}
\IEEEoverridecommandlockouts
% The preceding line is only needed to identify funding in the first footnote. If that is unneeded, please comment it out.
\usepackage{cite}
\usepackage{amsmath,amssymb,amsfonts}
\usepackage{algorithmic}
\usepackage{graphicx}
\usepackage{textcomp}
\usepackage{xcolor}
\def\BibTeX{{\rm B\kern-.05em{\sc i\kern-.025em b}\kern-.08em
    T\kern-.1667em\lower.7ex\hbox{E}\kern-.125emX}}
\begin{document}

\title{Upgrading the perfection, cube timer\\
\thanks{Identify applicable funding agency here. If none, delete this.}
}

\author{\IEEEauthorblockN{Miguel Andres Contreras Rodriguez }
\IEEEauthorblockA{\textit{Facultad Ingenieria}
\textit{Universidad Distrital FJC}\\
Bogota DC, Colombia \\
macontrerasr@udistrtital.edu.co}
\and
\IEEEauthorblockN{Santiago Andres Benavides Coral}
\IEEEauthorblockA{\textit{Facultad Ingenieria}
\textit{Universidad Distrital FJC}\\
Bogota DC, Colombia \\
sabenavidesc@udistrital.edu.co}
\and


}

\maketitle

\begin{abstract}
Solving Rubik cubes is something complex that needs specific tools for doing, we want to do a timer where take into account the main needs speed cubers have, 
\end{abstract}

\begin{IEEEkeywords}
Rubik's cube, timer, speed cubing
\end{IEEEkeywords}

\section{Introduction}

The difficult task of solving a Rubik’s Cube has evolved significantly over time. Enthusiasts dedicate countless hours to shaving off even 0.1 seconds from their personal bests. Numerous new techniques have emerged to achieve this challenging goal, and technology has advanced considerably. Modern cubes now feature magnets and redesigned cores, enhancing their performance and stability.
\\
\indent We can see the advance, the speed cubing has got on the WCA (World Cube Association), we can see that the world record for 3x3 has been beaten eleven times on the last ten years, but it only has decrease from 5.55 seconds to 3.13 seconds.
\\
\indent One crucial aspect of speed cubing is the timer. While a basic stopwatch on a cellphone or a quick Google search might suffice for casual timing, serious cubers require more sophisticated tools to track and improve their performance. A specialized timer can provide the precision and features necessary to aid in the journey towards faster solve times.
\\
\indent Just adding a few things to the timer, such as customize scrambles, session tracking, and statistical analysis, can make a significant difference. These features help cubers analyze their performance, identify areas for improvement, and track their progress over time. However, despite these advancements, timers in speed cubing still present several challenges.
\\\\
\centering 2. {METHOD AND MATERIALS}

\\\\
\indent One of the things we took into account was different type of cubes, it's better use an interface because this let us expand the application with more different type of cubes. 
\\
\indent We used the officisl API by WCA (World Cube Association) called Tnoodle, The reason why we used it is because it`s the most secure scramble for people who compite, like speedcubers or casual cubers.
\end{document}
